\documentclass[11pt,class=report,crop=false]{standalone}
\usepackage[screen]{../python}

\begin{document}


%====================================================================
\chapitre{Hello world!}
%====================================================================


\objectifs{Get into programming! In this very first activity, you will learn to manipulate numbers, variables and code your first loop with \Python.}




%%%%%%%%%%%%%%%%%%%%%%%%%%%%%%%%%%%%%%%%%%%%%%%%%%%%%%%%%%%%%%%%
%%%%%%%%%%%%%%%%%%%%%%%%%%%%%%%%%%%%%%%%%%%%%%%%%%%%%%%%%%%%%%%%

\begin{cours}[Numbers with \Python]

Check that \Python{} works correctly, by typing the following commands in the  \Python{} console:
\begin{lstlisting}
>>> 2+2
>>> "Hello world!"
\end{lstlisting}

Here are some instructions to try.
\begin{itemize}
  \item \textbf{Addition.} \ci{5+7}.
  \item \textbf{Multiplication.} 
  \ci{6*7}; with brackets \ci{3*(12+5)}; with decimal numbers \ci{3*1.5}.
  \item \textbf{Power.}\index{power}\index{**@\ci{**}} \ci{3**2} for $3^2=9$; negative power \ci{10**-3} for $10^{-3} = 0.001$.
  \item \textbf{Real division.} \ci{14/4} is equal to \ci{3.5}; \ci{1/3} is equal to \ci{0.3333333333333333}.
  \item \textbf{Integer division\index{Euclidean division} and modulo\index{modulo}\index{remainder}.}
  \begin{itemize}
    \item \ci{14//4} returns \ci{3}: it is the quotient of the Euclidean division of $14$ by $4$, note the double slash;
    \item \ci{14\%4} returns \ci{2}: it is the remainder of the Euclidean division of $14$ by $4$, we also say \og{}$14$ modulo $4$\fg.
  \end{itemize}
\end{itemize}

\emph{Note.} Inside the computer, decimals numbers are encoded as \og{}floating point numbers\fg{}\index{floating point number}.
\end{cours}


%%%%%%%%%%%%%%%%%%%%%%%%%%%%%%%%%%%%%%%%%%%%%%%%%%%%%%%%%%%%%%%%
% Activity 1
%%%%%%%%%%%%%%%%%%%%%%%%%%%%%%%%%%%%%%%%%%%%%%%%%%%%%%%%%%%%%%%%

\begin{activite}[First steps]

\objectifs{Goal: code your first calculations with \Python.}

\begin{enumerate}
  \item How many seconds are there in a century? (Do not take leap years into account.)
  
  \item How far do you have to complete the dotted formula to obtain a number greater than one billion?
  $$(1+2)\times(3+4)\times(5+6)\times(7+8)\times\cdots$$
  
  \item What are the last three digits of 
  {\footnotesize
  $$123456789 \times 123456789 \times 123456789 \times 123456789 \times 123456789 \times 123456789 \times 123456789 \quad ?$$
  }
  \item $7$ is the first integer such that its inverse has a repeating decimal representation of period $6$:
  $$\frac{1}{7} = 0.\underbrace{142857}\underbrace{142857}\underbrace{142857}\ldots$$ 
  Find the first integer whose inverse has a repeating decimal representation of period $7$:
  $$\frac{1}{???} = 0.00\underbrace{abcdefg}\underbrace{abcdefg}\ldots$$
  
  \emph{Hint.} The integer is bigger than $230$!
  
  \item Find the unique integer:
  \begin{itemize}
    \item which gives a quotient of $107$ when you divide it by $11$ (with integer division),
    \item and which gives a quotient of $90$ when you divide it by $13$ (with integer division),
    \item and which gives a remainder equal to $6$ modulo $7$.
  \end{itemize}
    
\end{enumerate}

\end{activite}

%%%%%%%%%%%%%%%%%%%%%%%%%%%%%%%%%%%%%%%%%%%%%%%%%%%%%%%%%%%%%%%%
%%%%%%%%%%%%%%%%%%%%%%%%%%%%%%%%%%%%%%%%%%%%%%%%%%%%%%%%%%%%%%%%

\begin{cours}[Working with an editor]
From now on, it is better if you work in a text editor dedicated to \Python{} rather than with the console. You must then explicitly ask to display the result\index{print@\ci{print}}:
\begin{lstlisting}
print(2+2)
print("Hello world!")
\end{lstlisting}

In the following you continue to write your code in the editor but we will no longer indicate that you must use \ci{print()} to display the results.
\end{cours}



%%%%%%%%%%%%%%%%%%%%%%%%%%%%%%%%%%%%%%%%%%%%%%%%%%%%%%%%%%%%%%%%
%%%%%%%%%%%%%%%%%%%%%%%%%%%%%%%%%%%%%%%%%%%%%%%%%%%%%%%%%%%%%%%%

\begin{cours}[Variables]


\textbf{Variable.} 
A \defi{variable}\index{variable} is a name associated with a memory location. It is like a box that is identified by a label.
The command \og{}\ci{a = 3}\fg{}\index{=@\ci{=}}\index{assignment} means that I have a variable \og{}\ci{a}\fg{} associated with the value $3$. 

Here is a first example:
\begin{lstlisting}
a = 3  # One variable
b = 5  # Another variable

print("The sum is",a+b)   # Display the sum
print("The product",a*b)  # Display the product

c = b**a     # New variable...
print(c)     # ... that is displayed
\end{lstlisting}


\medskip

\textbf{Comments.}
\index{\#@\#}
\index{comments}
Any text following the hashtag character \og{}\#\fg{} is not executed by \Python{} but is used to explain the program. It is a good habit to comment extensively on your code.

 
\medskip

\textbf{Names.} 
It is very important to give a clear and precise name to the variables. For example, with the right names you should know what the following code calculates:
\begin{lstlisting}
base = 8
height = 3
area = base * height / 2
print(area)
# print(Area)  # !! Error !!
\end{lstlisting}

Attention! \Python{} is case sensitive. So \ci{myvariable}, \ci{Myvariable} and \ci{MYVARIABLE} are different variables.

\medskip

\textbf{Re-assignment.}\index{assignment} Imagine you want to keep your daily accounts. You start with $S_0 = 1000$, the next day you earn $100$, so now $S_1 = S_0 + 100$; the next day you add $200$, so $S_2 = S_1 + 200$; then you lose $50$, so on the third day $S_3 = S_2 - 50$. With \Python{} you can use just one variable \ci{S} for all these operations.

\begin{lstlisting}
S = 1000
S = S + 100
S = S + 200
S = S - 50
print(S)
\end{lstlisting}
You have to understand the instruction \og{}\ci{S = S + 100}\fg{} like this: \og{}I take the contents of the box \ci{S}, I add $100$, I put everything back in the same box\fg{}.
\end{cours}


%%%%%%%%%%%%%%%%%%%%%%%%%%%%%%%%%%%%%%%%%%%%%%%%%%%%%%%%%%%%%%%%
% Activity 2
%%%%%%%%%%%%%%%%%%%%%%%%%%%%%%%%%%%%%%%%%%%%%%%%%%%%%%%%%%%%%%%%

\begin{activite}[Variables]

\objectifs{Goal: use variables!}

\begin{enumerate}
  \item 
  \begin{enumerate}
    \item Define variables, then calculate the area of a trapezoid. 
  Your program should display \ci{"The value of the area is ..."} using \ci{print("The value of the area is",area)}.

\myfigure{0.9}{
  \tikzinput{fig_trapezoid}
}  

    
    \item Define variables to calculate the volume of a box (a rectangular parallelepiped) whose dimensions are $10$, $8$, $3$.
    
    \item Define a variable \ci{PI} equals to $3.14$. Define a radius $R = 10$. Write the formula for the area of a disc of radius $R$.

\end{enumerate}    

  
  \item Put the lines back in order so that, at the end, $x$ has the value $46$.
\begin{center}
\begin{minipage}{0.5\textwidth}
\begin{lstlisting}  
(1)  y = y - 1
(2)  y = 2*x
(3)  x = x + 3*y     
(4)  x = 7
\end{lstlisting}
\end{minipage}
\end{center}  
  
  \item You place the sum of $1000$ dollars in a savings account. Each year the interest on the money invested brings in $10\%$ (the capital is multiplied by $1.10$).
  Write the code to calculate the capital for the first three years.

  \item I define two variables by \ci{a = 9} and \ci{b = 11}. I would like to exchange the content of \ci{a} and \ci{b}. Which instructions should I use so that at the end \ci{a} equals $11$ and \ci{b} equals $9$?
  
\begin{center}
\begin{minipage}{0.2\textwidth}
\begin{lstlisting}  
a = b
b = a   
\end{lstlisting}
\end{minipage}
\begin{minipage}{0.2\textwidth}
\begin{lstlisting}  
c = b
a = b
b = c
\end{lstlisting}
\end{minipage}
\begin{minipage}{0.2\textwidth}
\begin{lstlisting}
c = a
a = b
b = c 
\end{lstlisting}
\end{minipage}
\begin{minipage}{0.2\textwidth}
\begin{lstlisting} 
c = a
a = c
c = b
b = c
\end{lstlisting}
\end{minipage}

\end{center} 
 
\end{enumerate}  
\end{activite}


%%%%%%%%%%%%%%%%%%%%%%%%%%%%%%%%%%%%%%%%%%%%%%%%%%%%%%%%%%%%%%%%
%%%%%%%%%%%%%%%%%%%%%%%%%%%%%%%%%%%%%%%%%%%%%%%%%%%%%%%%%%%%%%%%

\begin{cours}[Use functions]
\sauteligne
\begin{itemize}
  \item \textbf{Use \Python{} functions.}
  
  You already know the \ci{print()} function that displays a string of characters (or numbers). It can be used by \ci{print("Hi there.")} or through a value: 
\begin{center}
\begin{minipage}{0.5\textwidth}
\begin{lstlisting}  
string = "Hi there."
print(string)
\end{lstlisting}
\end{minipage}
\end{center} 	 

  There are many other functions. For example, the function \ci{abs()} calculates the absolute value of a number: for example \ci{abs(-3)} returns \ci{3}, \ci{abs(5)} returns \ci{5}.
 % Other example: \ci{float()} transforms a character string into a number: \ci{float("+1.234567")} returns the floating number \ci{1.234567}.

  \item \textbf{The module \ci{math}.}

	Not all functions are directly accessible. They are often grouped into \defi{modules}\index{module}. For example, the \ci{math} module\index{math@\ci{math}}\index{module!math@\ci{math}} contains mathematical functions. For instance, you will find the square root function \ci{sqrt()}\index{sqrt@\ci{sqrt}}\index{square root}. Here's how to use it: 
\begin{center}
\begin{minipage}{0.5\textwidth}
\begin{lstlisting} 
from math import *

x = sqrt(2)
print(x)
print(x**2)
\end{lstlisting}
\end{minipage}
\end{center} 	

The first line imports all the functions of the module named \ci{math}\index{import@\ci{import}}, the next lines calculate $x = \sqrt{2}$ (as an approximate value) and then display $x$ and $x^2$.	

  \item \textbf{Sine and cosine.} 
  
  The \ci{math} module contains the trigonometric functions sine \index{sin@\ci{sin}} and cosine \index{cos@\ci{cos}} and even the constant \ci{pi}\index{pi@\ci{pi}}\index{$\pi$} which is an approximate value of $\pi$. Be careful, the angles are expressed in radians.

Here is the calculation of $\sin(\frac\pi2)$. 
\begin{center}
\begin{minipage}{0.5\textwidth} 
\begin{lstlisting}
angle = pi/2
print(angle)
print(sin(angle))
\end{lstlisting}
\end{minipage}
\end{center}

  \item \textbf{Decimal to integer.}

In the \ci{math} module there are also functions to round a decimal number: 
  \begin{itemize}
    \item \ci{round()}\index{round@\ci{round}} rounds to the nearest integer: \ci{round(5.6)} returns \ci{6}, \ci{round(1.5)} returns \ci{2}.
    \item \ci{floor()}\index{floor@\ci{floor}} returns the integer less than or equal to: \ci{floor(5.6)} returns \ci{5}.
    \item \ci{ceil()}\index{this@\ci{ceil}} returns the integer greater than or equal to: \ci{ceil(5.6)} returns \ci{6}.
  \end{itemize}       
    
\end{itemize}

\end{cours}

\bigskip

\bigskip

\bigskip


%%%%%%%%%%%%%%%%%%%%%%%%%%%%%%%%%%%%%%%%%%%%%%%%%%%%%%%%%%%%%%%%
% Activity 3
%%%%%%%%%%%%%%%%%%%%%%%%%%%%%%%%%%%%%%%%%%%%%%%%%%%%%%%%%%%%%%%%

\begin{activite}[Use functions]

\objectifs{Goal: use functions from \Python{} and the \ci{math} module.}

\begin{enumerate}
  \item The \Python{} function for gcd is \ci{gcd(a,b)}\index{gcd@\ci{gcd}} (for greatest common divisor). Calculate the gcd of $a = 10\,403$ and $b = 10\,506$. Deduce the lcm\index{lcm} from $a$ and $b$. The function lcm does not exist, you must use the formula:
  $$\text{lcm}(a,b) = \frac{a \times b}{\text{gcd}(a,b)}.$$
  
  \item By trial and error, find a real number $x$ that checks all the following conditions (several solutions are possible):
  \begin{itemize}
    \item \ci{abs(x**2 - 15)} is less than \ci{0.5}
    \item \ci{round(2*x)} returns \ci{8}
    \item \ci{floor(3*x)} returns \ci{11}
    \item \ci{ceil(4*x)} returns \ci{16} 
  \end{itemize}
 
  \emph{Hint.} \ci{abs()}\index{abs@\ci{abs}}\index{absolute value} refers to the absolute value function.
  
  \item You know the trigonometric formula 
  $$\cos^2 \theta + \sin^2 \theta = 1.$$
  Check that for $\theta = \frac\pi7$ (or other values) this formula is numerically true (this is not a proof of the formula, because \Python{} only makes approximate computations of the sine and cosine).
\end{enumerate}  
\end{activite}


%%%%%%%%%%%%%%%%%%%%%%%%%%%%%%%%%%%%%%%%%%%%%%%%%%%%%%%%%%%%%%%%
%%%%%%%%%%%%%%%%%%%%%%%%%%%%%%%%%%%%%%%%%%%%%%%%%%%%%%%%%%%%%%%%


\begin{cours}[\og{}for\fg{} loop]

\objectifs{The \og{}for\fg{} loop is the easiest way to repeat instructions.}

\index{for@\ci{for}}
\index{in@\ci{in}}
\index{loop!for}

\vspace*{1ex}

\mybox{
\myfigure{0.70}{
  \tikzinput{fig_hello_world_loop_for}
} }


Note that what delimits the block of instructions to be repeated is \defi{indentation}\index{indentation}, i.e.~the spaces at the beginning of each line that shift the lines to the right.
All lines in a block must have exactly the same indentation. In this book, we choose an indentation of $4$ spaces.

Don't forget the colon \og{}\ci{:}\fg{} at the end of the line of the \ci{for} declaration!

\begin{itemize}
  \item \textbf{Example of a \og{}for\fg{} loop.}

Here is a loop that displays the squares of the first integers. 
\begin{center}
\begin{minipage}{0.5\textwidth} 
\begin{lstlisting}
for i in range(10):
    print(i*i)
\end{lstlisting}
\end{minipage}
\end{center} 
The second line is shifted and constitutes the block to be repeated.
The variable \ci{i} takes the value $0$ and the instruction displays $0^2$;
then \ci{i} takes the value $1$, and the instruction displays $1^2$; then $2^2$, $3^2$\ldots

In the end this program displays:
$$0,1,4,9,16,25,36,49,64,81.$$

Warning: the last value taken by \ci{i} is $9$ (and not $10$).

  \item \textbf{Browse any list.}
  
The \og{}for\fg{} loop allows you to browse any list. Here is a loop that displays the cube of the first prime numbers.
\begin{center}
\begin{minipage}{0.5\textwidth} 
\begin{lstlisting}
for p in [2,3,5,7,11,13]:
    print(p**3)
\end{lstlisting}
\end{minipage}
\end{center} 

  \item \textbf{Sum all.}

Here is a program that calculates 
$$0+1+2+3+\cdots + 18 +19.$$

\begin{center}
\begin{minipage}{0.5\textwidth} 
\begin{lstlisting}
mysum = 0
for i in range(20):
    mysum = mysum + i
print(mysum)
\end{lstlisting}
\end{minipage}
\end{center} 

Understand this code well: a variable \ci{mysum} is initialized at $0$. 
We will add $0$, then $1$, then $2$\ldots{}
This loop can be better understood by filling in a table:
  $$
  \begin{array}{l}
  \text{Initialisation: \ci{mysum}$=0$}    \\
  \begin{array}{|c|c|}
  \hline  
   \text{\ci{i}} &  \text{\ci{mysum}} \\
  \hline\hline
  0 & 0 \\  
  1 & 1 \\
  2 & 3 \\
  3 & 6 \\
  4 & 10\\
  \hspace*{5ex}\ldots\hspace*{5ex} & \hspace*{5ex}\ldots\hspace*{5ex} \\
  18 & 171 \\
  19 & 190 \\ 
  \hline
  \end{array} \\
  \text{Display: $190$}  
  \end{array} 
  $$ 
\end{itemize} 

% No index inside a mybox!  

\index{range@\ci{range}}
\index{list@\ci{list}}  
\mybox
{ 
 \begin{minipage}{0.95\textwidth}
 \begin{itemize}
 \item \textbf{\ci{range()}.}
	\begin{itemize}
	  \item With \ci{range(n)} we run the entire range from $0$ to $n-1$.
	  For example \ci{range(10)} corresponds to the list \ci{[0, 1, 2, 3, 4, 5, 6, 7, 8, 9]}. 
	  
	  Attention! the list stops at $n-1$ and not at $n$. What to remember 
	  is that the list contains $n$ items (because it starts at $0$).
	  
	 \item If you want to display the list of items browsed, you must use the command:	 
	  \mycenterline{\ci{list(range(10))}}
	
	\item With \ci{range(a,b)} we go through the elements from $a$ to $b-1$.
	For example \ci{range(10,20)} corresponds to the list \ci{[10, 11, 12, 13, 14, 15, 16, 17, 18, 19]}.  
	
	\item With \ci{range(a,b,step)} you can browse the items $a$, $a+\text{step}$, $a + 2\text{step}$\ldots{} For example \ci{range(10,20,2)} corresponds to the list \ci{[10, 12, 14, 16, 18]}.  

	\end{itemize}
\end{itemize}
\end{minipage}	
}
\begin{itemize}
  \item \textbf{Nested loops.}

It is possible to nest loops, i.e.~use a loop inside the block of another loop.
\begin{center}
\begin{minipage}{0.5\textwidth} 
\begin{lstlisting}
for x in [10,20,30,40,50]:
    for y in [3,7]:
        print(x+y)
\end{lstlisting}
\end{minipage}
\end{center}
In this small program $x$ is first equal to $10$, $y$ takes the value $3$ then the value $7$ (so the program displays $13$, then $17$). Then $x=20$, and $y$ again equals $3$, then $7$ again (so the program displays $23$, then $27$). 
Finally the program displays:
$$13,17,23,27,33,37,43,47,53,57.$$
    
\end{itemize} 
\end{cours}


%%%%%%%%%%%%%%%%%%%%%%%%%%%%%%%%%%%%%%%%%%%%%%%%%%%%%%%%%%%%%%%%
% Activity 4
%%%%%%%%%%%%%%%%%%%%%%%%%%%%%%%%%%%%%%%%%%%%%%%%%%%%%%%%%%%%%%%%

\begin{activite}[\og{}for\fg{} loop]

\objectifs{Goal: build simple loops.}

\begin{enumerate}
  \item 
  \begin{enumerate}
    \item Display the cubes of integers from $0$ to $100$.
    \item Display the fourth powers of integers from $10$ to $20$.
    \item Display the square roots of integers $0$, $5$, $10$, $15$,\ldots{} up to $100$.   
  \end{enumerate}

  \item Display the powers of $2$, from $2^1$ to $2^{10}$, and memorize the results!
  
  
  \item Experimentally search for a value close to the minimum of the function 
  $$f(x) = x^3-x^2-\frac14x+1$$
   on the interval $[0,1]$.
  
  \emph{Hints.}
  \begin{itemize}
    \item Build a loop in which a variable $i$ scans integers from $0$ to $100$.
    \item Defined $x=\frac{i}{100}$. So $x=0.00$, then $x=0.01$, $x=0.02$\ldots
    \item Calculate $y = x^3-x^2-\frac14x+1$.
    \item Display the values using \ci{print("x =",x,"y =",y)}.
    \item Search by hand for which value of $x$ you get the smallest possible $y$.
    \item Feel free to modify your program to increase accuracy.
  \end{itemize}
    
  \item Seek an approximate value that must have
  the radius $R$ of a ball so that its volume is $100$.
  
  \emph{Hints.} 
  \begin{itemize}
    \item Use a scanning method as in the previous question.
    \item The formula for the volume of a ball is $V = \frac43 \pi R^3$.
    \item Display values using \ci{print("R =",R,"V =",V)}.    
    \item For $\pi$ you can take the approximate value $3.14$ or the approximate value \ci{pi} of the \ci{math} module.
  \end{itemize}
  
\end{enumerate}  
\end{activite}


%%%%%%%%%%%%%%%%%%%%%%%%%%%%%%%%%%%%%%%%%%%%%%%%%%%%%%%%%%%%%%%%
% Activity 5
%%%%%%%%%%%%%%%%%%%%%%%%%%%%%%%%%%%%%%%%%%%%%%%%%%%%%%%%%%%%%%%%

\begin{activite}[\og{}for\fg{} loop (continued)]

\objectifs{Goal: build more complicated loops.}

\begin{enumerate}
  \item Define a variable $n$ (for example $n=20$). Calculate the sum
  $$1^2+2^2+3^2+\cdots+i^2+\cdots +n^2.$$
  
  \item Calculate the product:
  $$1 \times 3 \times 5 \times \cdots \times 19.$$
  
  \emph{Hints.} Begin by defining a \ci{myproduct} variable initialized to the value $1$. Use \ci{range(a,b,2)} to get every other integer.
  
  \item Display multiplication tables between $1$ and $10$. Here is an example of a line to display:
  \mycenterline{\ci{7 x 9 = 63}}
  
  Use a display command of the style: \ci{print(a,"x",b,"=",a*b)}.
\end{enumerate}  
\end{activite}

\end{document}


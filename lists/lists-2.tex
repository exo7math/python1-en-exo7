\documentclass[11pt,class=report,crop=false]{standalone}
\usepackage[screen]{../python}
\begin{document}

%====================================================================
\chapitre{Lists II}
%====================================================================

\objectifs{The lists are so useful that you have to know how to handle them in a simple and efficient way. That's the purpose of this chapter!}

\index{list}

%%%%%%%%%%%%%%%%%%%%%%%%%%%%%%%%%%%%%%%%%%%%%%%%%%%%%%%%%%%%%%%%
%%%%%%%%%%%%%%%%%%%%%%%%%%%%%%%%%%%%%%%%%%%%%%%%%%%%%%%%%%%%%%%%

\begin{cours}[Manipulate lists efficiently]
\sauteligne
\begin{itemize}
  \item \textbf{Slicing lists.}
  
  \index{list!slice}

  \begin{itemize}
    \item You already know \ci{mylist[a:b]} that returns the sublist of elements from the rank $a$ to the rank $b-1$.
    
    \item \ci{mylist[a:]} returns the list of elements from rank $a$ until the end.
      
    \item \ci{mylist[:b]} returns the list of elements from the beginning to the rank $b-1$.
    
    \item \ci{mylist[-1]} returns the last element, \ci{mylist[-2]} returns the penultimate element, \ldots

  \item \textbf{Exercise.} 
  \myfigure{0.7}{
  \tikzinput{fig-lists-4}
}   
  
  With \ci{mylist = [7,2,4,5,3,10,9,8,3]}, 
what do the following instructions return?

\begin{itemize}
  \item \ci{mylist[3:5]}
  \item \ci{mylist[4:]}
  \item \ci{mylist[:6]}
  \item \ci{mylist[-1]}
\end{itemize}

   \end{itemize} 
   
  \item \textbf{Find the rank of an element.} 

\begin{itemize}

    \item \index{index@\ci{index}}
   \ci{mylist.index(element)} returns the first position at which the item was found. Example: with \ci{mylist = [12, 30, 5, 9, 5, 21]},
   \ci{mylist.index(5)} returns $2$.

  \item \index{in@\ci{in}} 
  If you just want to know if an item belongs to a list, then the statement:
  \mycenterline{\ci{element in mylist}}  
  
  returns \ci{True} or \ci{False}.
  Example: with \ci{mylist = [12, 30, 5, 9, 5, 21]},
   \og\ci{9 in mylist}\fg{} is true, while \og\ci{8 in mylist}\fg{} is false.
  
\end{itemize}
   
  \item \textbf{List comprehension.}
  
  \index{list!comprehension}
  
  A set can be defined by listing all its elements, for example $E = \{0,2,4,6,8,10\}$. Another way is to say that the elements of the set must verify a certain property. For example, the same set $E$ can be defined by:
  $$E = \{ x \in \Nn \mid x \le 10 \text{ and } x \text{ is even} \}.$$
  
  With \Python{} there is a way to define lists this way. It is an extremely powerful and efficient syntax. Let's look at some examples:
  \begin{itemize}
    \item Let's start from a list, for example \ci{mylist = [1,2,3,4,5,6,7,6,5,4,3,2,1]}.
    
    \item The command \ci{mylist_doubles = [ 2*x for x in mylist ]} returns a list that contains the double of each item of the \ci{mylist} list. So this is the list \ci{[2,4,6,8,...]}.
    
    \item The command \ci{mylist_squares = [ x**2 for x in mylist ]} returns the list of squares of the items in the initial list. So this is the list \ci{[1,4,9,16,...]}.
    
    \item The command \ci{mylist_partial = [x for x in mylist if x > 2]}
    extracts the list composed only of elements greater than $2$. So this is the list \ci{[3,4,5,6,7,6,5,4,3]}.
	\end{itemize}
	
	 
  
  \item \textbf{List of lists.}
  
  \index{list!of lists}
  \index{array}
  
  A list can contain other lists, for example:  
  \mycenterline{\ci{mylist = [ ["Harry", "Hermione", "Ron"], [101,103] ]}}
  
   contains two lists. 
  We will be interested in lists that contain lists of integers, which we will call \defi{arrays}. For example:   
  \mycenterline{\ci{array = [ [2,14,5], [3,5,7], [15,19,4], [8,6,5] ]}}
  
  Then \ci{array[i]} returns the sublist of index $i$, while
  \ci{array[i][j]} returns the integer located at the index $j$ in the sublist of index $i$. For example:
  \begin{itemize}
  \item \ci{array[0]} returns the list \ci{[2,14,5]},
  \item \ci{array[1]} returns the list \ci{[3,5,7]},
  \item \ci{array[0][0]} returns the integer \ci{2},
  \item \ci{array[0][1]} returns the integer \ci{14},
  \item \ci{array[2][1]} returns the integer \ci{19}.
\end{itemize}

\end{itemize}
\end{cours}


%%%%%%%%%%%%%%%%%%%%%%%%%%%%%%%%%%%%%%%%%%%%%%%%%%%%%%%%%%%%%%%%
% Activity 1
%%%%%%%%%%%%%%%%%%%%%%%%%%%%%%%%%%%%%%%%%%%%%%%%%%%%%%%%%%%%%%%%

\begin{activite}[Lists comprehension]

\objectifs{Goal: practice list comprehension. In this activity the lists are lists of integers.}

\begin{enumerate}
  \item Program a \ci{multiplication(mylist,k)} function that multiplies each item in the list by $k$. For example, \ci{multiplication([1,2,3,4,5],2)} returns \ci{[2,4,6,8,10]}.
  
  \item Program a \ci{power(mylist,k) function} that raises each element of the list to the power $k$. For example, \ci{power([1,2,3,4,5],3)} returns \ci{[1,8,27,64,125]}.
  
  \item Program an \ci{addition(mylist1,mylist2)} function that adds together the elements of two lists of the same length. For example, \ci{addition([1,2,3],[4,5,6])} returns \ci{[5,7,9]}.
  
  \emph{Hint.} This is an example of a task where list comprehension is not used!
  

  \item Program a \ci{non_zero(mylist)} function that returns a list of all non-zero elements. For example, \ci{non_zero([1,0,2,3,0,4,5,0])} returns \ci{[1,2,3,4,5]}.
  
  \item Program an \ci{even(mylist)} function that returns a list of all even elements. For example, \ci{even([1,0,2,3,0,4,5,0])} returns \ci{[0,2,0,4,0]}.
  
\end{enumerate}

\end{activite}



%%%%%%%%%%%%%%%%%%%%%%%%%%%%%%%%%%%%%%%%%%%%%%%%%%%%%%%%%%%%%%%%
% Activity 2
%%%%%%%%%%%%%%%%%%%%%%%%%%%%%%%%%%%%%%%%%%%%%%%%%%%%%%%%%%%%%%%%

\begin{activite}[Reach a fixed amount]

\objectifs{Goal: try to reach the total of $100$ in a list of numbers.}

We consider a list of $n$ integers between $1$ and $99$ (included).
For example, this list of 25 integers:
\mycenterline{\ci{[16,2,85,27,9,45,98,73,12,26,46,25,26,49,18,99,10,86,7,42]}}
which was obtained at random by the command:
\mycenterline{\ci{mylist_20 = [randint(1,99) for i in range(20)]}}

We are looking for different ways to find numbers from the list whose sum is exactly $100$.

\begin{enumerate}
  \item Program a \ci{sum_twoinarow_100(mylist)} function that tests if there are two consecutive elements in the list whose sum is exactly $100$. The function returns \ci{True} or \ci{False} (but it can also display numbers and their position for verification). For the example given, this function returns \ci{False}.
  
  \item Program a \ci{sum_two_100(mylist)} function that tests if there are two items in the list, located at different positions, whose sum is equal to $100$. 
 For the example given, this function returns \ci{True} and can display the integers $2$ and $98$ (at ranks $1$ and $6$ of the list). 
 
 
  \item Program a \ci{sum_seq_100(mylist)} function that tests if there are consecutive elements in the list whose sum is equal to $100$. 
 For the example given, this function returns \ci{True} and can display the sequence of integers $25$, $26$, $49$ (at ranks $11$, $12$ and $13$).
 
 \item \emph{(Optional.)} The larger the size of the list, the more likely it is to have values in the list whose sum is $100$. For each of the three previous situations, determine the size $n$ of the list such that the probability of obtaining a sum of $100$ is greater than $1/2$. 
 
  \emph{Hints.} For each case, you can get an estimate of this integer $n$, by writing a \ci{proba(n,N)} function that performs a large number $N$ of random draws of lists having $n$ items ($N=10\,000$ for example). The probability is approximated by the number of successful cases (where the function returns \ci{True}) divided by the total number of cases (here $N$).
 
 
  
\end{enumerate}

\end{activite}



%%%%%%%%%%%%%%%%%%%%%%%%%%%%%%%%%%%%%%%%%%%%%%%%%%%%%%%%%%%%%%%%
% Activity 3
%%%%%%%%%%%%%%%%%%%%%%%%%%%%%%%%%%%%%%%%%%%%%%%%%%%%%%%%%%%%%%%%

\begin{activite}[Arrays]

\objectifs{Goal: working with lists of lists. }

In this activity we work with arrays of size $n \times n$ containing integers.
The object \ci{array} is therefore a list of $n$ lists, each having $n$ elements.

For example ($n=3$): 
\mycenterline{\ci{array = [ [1,2,3], [4,5,6], [7,8,9] ]}}
represents the array:
$$\begin{array}{ccc}1&2&3\\4&5&6\\7&8&9\end{array}$$

\begin{enumerate}
  \item Write a \ci{sum_diagonal(array)} function that calculates the sum of the elements located on the main diagonal of an array.
  The main diagonal of the example given is $1$, $5$, $9$, so the sum is $15$.
  
  \item Write a \ci{sum_antidiagonal(array)} function that calculates the sum of the elements located on the other diagonal.
  The anti-diagonal of the example given is composed of $3$, $5$, $7$, the sum is still $15$.
  
  \item Write a \ci{sum_all(array)} function that calculates the total sum of all elements. In this example the total sum is $45$.
  
  \item Write a \ci{print_array(array)} function that displays an array properly on the screen. You can use the command:
\mycenterline{\ci{print('\{:>3d\}'.format(array[i][j]), end="")}}  

\emph{Explanations.}
\begin{itemize}
  \item The command \ci{print(string,end="")} allows you to display a string of characters without going to the next line.
  
  \item The command \ci{'\{:>3d\}'.format(k)} displays the integer $k$ on three characters (even if there is only one digit to display).
\end{itemize}  
\end{enumerate}

\end{activite}



%%%%%%%%%%%%%%%%%%%%%%%%%%%%%%%%%%%%%%%%%%%%%%%%%%%%%%%%%%%%%%%%
% Activity 4
%%%%%%%%%%%%%%%%%%%%%%%%%%%%%%%%%%%%%%%%%%%%%%%%%%%%%%%%%%%%%%%%

\begin{activite}[Magic Squares]

\index{Magic square@magic square}

\objectifs{Goal: build magic squares as big as you want! You must first have done the previous activity.}


A \defi{magic square} is a square array of size $n\times n$ that contains all the integers from $1$ to $n^2$ and satisfies that:
the sum of each row, the sum of each column, the sum of the main diagonal and the sum of the anti-diagonal all have the same value.

Here is an example of a magic square with a size of $3\times 3$ and one of size  $4\times 4$.


  \myfigure{1}{
  \tikzinput{fig-lists-5}
}  

%square_3x3 = [ [[4,9,2],[3,5,7],[8,1,6] ]
%square_4x4 = [ [1,14,15,4],[7,9,6,12],[10,8,11,5],[16,3,2,13] ]


For a magic square of size $n \times n$, the value of the sum is:
$$S_n = \frac{n(n^2+1)}{2}.$$



\begin{enumerate}
  \item \textbf{Examples.} Define an array for each of the $3 \times 3$ and $4 \times 4$ examples above and display them on the screen (use the previous activity).
  
  \item \textbf{To be or not to be.} Define an \ci{is_magic_square(square)} function that tests whether a given array is (or isn't) a magic square (use the previous activity for diagonals).
  
  \item \textbf{Random squares.} \emph{(Optional.)} Randomly generate squares containing integers from $1$ to $n^2$ using a \ci{random_square(n)} function. Experimentally verify that it is rare to obtain a magic square in this way! 

  \emph{Hints.} For a list \ci{mylist}, the command \ci{shuffle(mylist)}\index{shuffle@\ci{shuffle}} (from the \ci{random} module) randomly mixes the list (the list is modified in place). 
 
 
 \medskip
	
  \emph{The purpose of the remaining questions is to create large magic squares.}
	 
  
  \item \textbf{Addition.} Define an \ci{addition_square(square,k)} function
  which adds an integer $k$ to all the elements of the array. With the example of the $3\times 3$ square, the command \ci{addition_square(square,-1)} subtracts $1$ from all the elements and returns an array that would look like this:
$$\begin{array}{ccc}
3&8&1\\2&4&6\\7&0&5
\end{array}$$  

\emph{Hints.} To define a new square, start by filling it with $0$'s:
\mycenterline{\ci{new_square = [[0 for j in range(n)] for i in range(n)]}}
then fill it with the correct values using commands of the type:
\mycenterline{\ci{new_square[i][j] = ...}}

  \item \textbf{Multiplication.} Define a \ci{multiplication_square(square,k)} function
  which multiplies all the elements of the array by $k$. With the example of the  $3\times 3$ square, the \ci{multiplication_square(square,2)} command multiplies all the elements by $2$ and thus returns an array that would be displayed as follows:
$$\begin{array}{ccc}
8&18&4\\6&10&14\\16&2&12
\end{array}$$ 
  
  \item \textbf{Homothety.} Define a \ci{homothety_square(square,k)} function
  which enlarges the array by a factor of $k$ as shown in the examples below. 
  Here is an example of the $3 \times 3$ square with a homothety ratio of $k=3$.
 $$
  \begin{array}{c|c|c}  
  4& 9& 2\\\hline
  3& 5& 7\\\hline
  8& 1& 6\\  
  \end{array} 
\quad  \longrightarrow\quad
  \begin{array}{ccc|ccc|ccc}  
  4& 4& 4& 9& 9& 9& 2& 2& 2\\
  4& 4& 4& 9& 9& 9& 2& 2& 2\\ 
  4& 4& 4& 9& 9& 9& 2& 2& 2\\\hline
  3& 3& 3& 5& 5& 5& 7& 7& 7\\
  3& 3& 3& 5& 5& 5& 7& 7& 7\\
  3& 3& 3& 5& 5& 5& 7& 7& 7\\\hline
  8& 8& 8& 1& 1& 1& 6& 6& 6\\
  8& 8& 8& 1& 1& 1& 6& 6& 6\\
  8& 8& 8& 1& 1& 1& 6& 6& 6 \\
  \end{array}
$$
  
Here is an example of a $4\times 4$ square with a homothety ratio of $k=2$.
$$  
  \begin{array}{c|c|c|c}  
  1& 14& 15& 4\\\hline
  7& 9& 6& 12\\\hline
  10& 8& 11& 5\\\hline
  16& 3& 2& 13\\    
  \end{array} 
\quad  \longrightarrow \quad 
  \begin{array}{cc|cc|cc|cc}
 1& 1&14&14&15&15& 4& 4\\
  1& 1&14&14&15&15& 4& 4\\\hline
  7& 7& 9& 9& 6& 6&12&12\\
  7& 7& 9& 9& 6& 6&12&12\\\hline
 10&10& 8& 8&11&11& 5& 5\\
 10&10& 8& 8&11&11& 5& 5\\\hline
 16&16& 3& 3& 2& 2&13&13\\
 16&16& 3& 3& 2& 2&13&13
 \end{array}
  $$
  
\item \textbf{Block addition.} Define a \ci{block_addition_square(big_square,small_square)} function that adds a small array of size $n \times n$ to each block of the larger $nm \times nm$ sized array as shown in the example below with $n=2$ and $m=3$ (hence $nm=6$). The small $2 \times 2$ square on the left is added to the large square in the center to give the result on the right. For this addition the large square is divided into $9$ blocks, there is a total of $36$ additions.
$$
  \begin{array}{cc} 
  1& 2  \\
  3& 4  \\
  \end{array}
  \qquad
  \begin{array}{cc|cc|cc}  
  4& 4& 9& 9& 2& 2  \\
  4& 4& 9& 9& 2& 2  \\\hline
  3& 3& 5& 5& 7& 7  \\
  3& 3& 5& 5& 7& 7  \\\hline
  8& 8& 1& 1& 6& 6  \\
  8& 8& 1& 1& 6& 6  \\
  \end{array}
  \qquad  \longrightarrow \qquad
  \begin{array}{cc|cc|cc}
  5& 6&10&11& 3& 4  \\
  7& 8&12&13& 5& 6  \\\hline
  4& 5& 6& 7& 8& 9  \\
  6& 7& 8& 9&10&11  \\\hline
  9&10& 2& 3& 7& 8  \\
 11&12& 4& 5& 9&10  \\
  \end{array}
$$

 \item \textbf{Products of magic squares.} Define a \ci{product_squares(square1,square2)} function which from two magic squares, calculates a larger magic square called the product of the two squares. The algorithm is as follows:

 \begin{algorithme}
  \sauteligne 
 \begin{itemize}
   \item
   \begin{itemize}
     \item Inputs: a magic square $C_1$ of size $n\times n$ and a magic square $C_2$ of size $m\times m$.
     \item Output: a magic square $C$ of size $(nm)\times(nm)$.
   \end{itemize}

  \item Create square $C_{3a}$ by subtracting $1$ from all elements of $C_2$. (Use the \ci{addition_square(square2,-1)} command.)
  
  \item Define square $C_{3b}$ as the homothety of square $C_{3a}$ of ratio $n$. (Use the \ci{homothety(square3a,n) command}.)
  
  \item Define square $C_{3c}$ by multiplying all the terms of square $C_{3b}$ by $n^2$. (Use the \ci{multiplication_square(square3b,n**2)} command.)
  
  \item Define square $C_{3d}$ by adding square $C_1$ to square $C_{3c}$  block by block. (Use the \ci{block_addition_square(square3c,square1)} command.)
  
  \item Return the square $C_{3d}$.
   
 \end{itemize}  
 \end{algorithme}
 
 \begin{itemize}
   \item Implement this algorithm. 
   \item Test it on examples, checking that the square obtained is indeed a magic square.
   \item Build a magic square of size $36 \times 36$.
   \item Also confirm that the order of the product is important: $C_1 \times C_2$ is not the same square as $C_2 \times C_1$. 
  \end{itemize}  

\end{enumerate}

\end{activite}

\end{document}


\pagestyle{empty}\thispagestyle{empty}
\vspace*{\fill}
\vspace*{5ex}
\begin{center}
\fontsize{30}{30}\selectfont
\textsc{python in high school}

%\vspace*{-0.5ex}
%\textsc{\fontsize{24}{24}\selectfont volume \fontsize{22}{22}\selectfont 1}

\vspace*{2ex}

%\fontsize{32}{32}\selectfont
\Large
\textsc{arnaud bodin}

\end{center}
\vfill
\begin{center}
\Large
\textsc{algorithms \  and  \  mathematics}
\end{center}
\begin{center}
\LogoExoSept{2}
\end{center}

% \clearemptydoublepage
\clearpage

\thispagestyle{empty}

\vspace*{\fill}
%\section*{Python in high school -- volume 1}
\section*{Python in high school}

%---------------------------
{\large\textbf{Let's go!}}

Everyone uses a computer, but it's another thing to drive it! Here you will learn the basics of programming. The objective of this book is twofold: to deepen mathematics through computer science and to master programming with the help of mathematics. 

\bigskip

%---------------------------
{\large\textbf{Python}}

Choosing a programming language to start with is tricky. You need a language with easy handling, well documented, with a large community of users. Python has all these qualities and more. It is modern, powerful and widely used, including by professional programmers. 

Despite all these qualities, starting programming (with Python or another language) is difficult. The best thing is to already have experience with the code, using \emph{Scratch} for example. There are still big steps to climb and this book is here to accompany you.

\bigskip

%---------------------------
{\large\textbf{Objective}}

Mastering Python will allow you to easily learn other languages. Especially since the language is not the most important, the most important things are the algorithms. Algorithms are like cooking recipes, you have to follow the instructions step by step and what counts is the final result and not the language with which the recipe was written. This book is therefore neither a complete Python manual nor a computer course, nor is it about using Python as a super-calculator.

The aim is to discover algorithms, to learn step-by-step programming through mathematical/computer activities. This will allow you to put mathematics into practice with the willingness to limit yourself to the knowledge acquired during the first years.

\bigskip

%---------------------------
{\large\textbf{Mathematics for computer science}}\\
{\large\textbf{Computer science for mathematics}}

Since computers only handle numbers, mathematics is essential to communicate with them. Another example is the graphical on-screen display that requires a good understanding of the coordinates $(x,y)$, trigonometry\ldots{}.

Computers are a perfect match for mathematics! The computer becomes essential to manipulate very large numbers or to test conjecture on many cases. In this book you will discover fractals, L-systems, brownian trees and the beauty of complex mathematical phenomena.

\bigskip

\begin{center}
You can retrieve all the activity \Python{} codes and all the source files on the Exo7 \emph{GitHub} page:
\mycenterline{\href{https://github.com/exo7math/python1-en-exo7}{GitHub: Python in high school}}

\medskip

% The videos of the courses with step-by-step explanations and the presentation of the projects are available from the chain \emph{Youtube}:
%\href{https://www.youtube.com/channel/UC6PiFyqBiUjiJ7Q3DRSW2Wg}{\og{}Youtube : Python in high school{fg{}}.
\end{center}


\vspace*{\fill}



%\newpage
\cleardoublepage
\thispagestyle{empty}
\addtocontents{toc}{\protect\setcounter{tocdepth}{0}}
\tableofcontents


\newpage


\section*{Summary of the activities}


\newcommand{\titreactivite}[1]{{\textbf{#1}}\nopagebreak}
\newcommand{\descriptionactivite}[1]{%
\smallskip\hfill
\begin{minipage}{0.95\textwidth}\small#1\end{minipage}\medskip\smallskip}

%%%%%%%%%%%%%%%%%%%%%%%%%%%%%%%%%%%%%%%%%%%%%%%%%%%%
\titreactivite{Hello world!}

\descriptionactivite{Get into programming! In this very first activity, you will learn to manipulate numbers, variables and code your first loop with \Python.}

%%%%%%%%%%%%%%%%%%%%%%%%%%%%%%%%%%%%%%%%%%%%%%%%%%%%
\titreactivite{Turtle (Scratch with Python)}

\descriptionactivite{The \ci{turtle} module allows you to easily make drawings in \Python. It's about giving orders to a turtle with simple instructions like \og{}go ahead\fg{}, \og{}turn\fg{}\ldots{} It's the same principle as with \emph{Scratch}, but with one difference: you no longer move blocks, instead you write the instructions.}

%%%%%%%%%%%%%%%%%%%%%%%%%%%%%%%%%%%%%%%%%%%%%%%%%%%%
\titreactivite{If ... then ...}

\descriptionactivite{The computer can react according to a situation. 
If a condition is met, it acts in a certain way, otherwise it does something else.}

%%%%%%%%%%%%%%%%%%%%%%%%%%%%%%%%%%%%%%%%%%%%%%%%%%%%
\titreactivite{Functions}

\descriptionactivite{Writing a function is the easiest way to group code for a particular task, in order to execute it once or several times later.}

%%%%%%%%%%%%%%%%%%%%%%%%%%%%%%%%%%%%%%%%%%%%%%%%%%%%
\titreactivite{Arithmetic -- While loop -- I}

\descriptionactivite{The activities in this sheet focus on arithmetic: long division, prime numbers \ldots{} This is an opportunity to use the \og{}while\fg{} loop intensively.}

%%%%%%%%%%%%%%%%%%%%%%%%%%%%%%%%%%%%%%%%%%%%%%%%%%%%
\titreactivite{Strings -- Analysis of a text}

\descriptionactivite{You're going to do some fun activities by manipulating strings and characters.}

%%%%%%%%%%%%%%%%%%%%%%%%%%%%%%%%%%%%%%%%%%%%%%%%%%%%
\titreactivite{Lists I}

\descriptionactivite{A list is a way to group elements into a single object. After defining a list, you can retrieve each item of the list one by one, but also add new ones\ldots}


%%%%%%%%%%%%%%%%%%%%%%%%%%%%%%%%%%%%%%%%%%%%%%%%%%%%
\titreactivite{Statistics -- Data visualization}

\descriptionactivite{It's good to know how to calculate the minimum, maximum, average and quartiles of a series. It's even better to visualize them all on the same graph!}

%%%%%%%%%%%%%%%%%%%%%%%%%%%%%%%%%%%%%%%%%%%%%%%%%%%%
\titreactivite{Files}

\descriptionactivite{You will learn to read and write data with files.}

%%%%%%%%%%%%%%%%%%%%%%%%%%%%%%%%%%%%%%%%%%%%%%%%%%%%
\titreactivite{Arithmetic -- While loop -- II}

\descriptionactivite{Our study of numbers is further developed with the \og{}while\fg{} loop.}
% For this chapter you will need the \ci{is_prime()} function you wrote in the \og{}Arithmetic -- While loop -- I\fg{} part.}

%%%%%%%%%%%%%%%%%%%%%%%%%%%%%%%%%%%%%%%%%%%%%%%%%%%%
\titreactivite{Binary I}

\descriptionactivite{The computers transform all data into numbers and manipulate only those numbers. These numbers are stored in the form of lists of $1$'s and $0$'s. It's the binary numeral system of numbers. To better understand this binary numeral system, you will first need to understand the decimal numeral system better.}

%%%%%%%%%%%%%%%%%%%%%%%%%%%%%%%%%%%%%%%%%%%%%%%%%%%%
\titreactivite{Lists II}

\descriptionactivite{The lists are so useful that you have to know how to handle them in a simple and efficient way. That's the purpose of this chapter!}

%%%%%%%%%%%%%%%%%%%%%%%%%%%%%%%%%%%%%%%%%%%%%%%%%%%%
\titreactivite{Binary II}

\descriptionactivite{We continue our exploration of the world of $1$'s and $0$'s.}

%%%%%%%%%%%%%%%%%%%%%%%%%%%%%%%%%%%%%%%%%%%%%%%%%%%%
\titreactivite{Probabilities -- Parrondo's paradox}

\descriptionactivite{You will program two simple games. When you play these games, you are more likely to lose than to win. However, when you play both games at the same time, you have a better chance of winning than losing! It's a paradoxical situation.}

%%%%%%%%%%%%%%%%%%%%%%%%%%%%%%%%%%%%%%%%%%%%%%%%%%%%
\titreactivite{Find and replace}

\descriptionactivite{Finding and replacing are two very frequent tasks. Knowing how to use them and how they work will help you to be more effective.}

%%%%%%%%%%%%%%%%%%%%%%%%%%%%%%%%%%%%%%%%%%%%%%%%%%%%
\titreactivite{Polish calculator -- Stacks}

\descriptionactivite{You're going to program your own calculator! For that you will discover a new notation for formulas and also discover what a \og{}stack\fg{} is in computer science.}

%%%%%%%%%%%%%%%%%%%%%%%%%%%%%%%%%%%%%%%%%%%%%%%%%%%%
\titreactivite{Text viewer -- Markdown}

\descriptionactivite{You will program a simple word processor that displays paragraphs cleanly and highlights words in bold and italics.}

%%%%%%%%%%%%%%%%%%%%%%%%%%%%%%%%%%%%%%%%%%%%%%%%%%%%
\titreactivite{L-systems}

\descriptionactivite{L-systems offer a very simple way to code complex phenomena. From an initial word and a number of replacement operations, we arrive at complicated words. When you \og{}draw\fg{} these words, you get beautiful fractal figures. The \og{}L\fg{} comes from the botanist A.~Lindenmayer who invented L-systems to model plants.}

%%%%%%%%%%%%%%%%%%%%%%%%%%%%%%%%%%%%%%%%%%%%%%%%%%%%
\titreactivite{Dynamic images}

\descriptionactivite{We will distort images. By repeating these distortions, the images become blurred. But by a miracle after a certain number of repetitions the original image reappears!}

%%%%%%%%%%%%%%%%%%%%%%%%%%%%%%%%%%%%%%%%%%%%%%%%%%%%
\titreactivite{Game of life}

\descriptionactivite{The \emph{game of life} is a simple model of the evolution of a population of cells that split and die over time. The \og{}game\fg{} consists of finding initial configurations that give interesting evolution: some groups of cells disappear, others stabilize, some move\ldots}

%%%%%%%%%%%%%%%%%%%%%%%%%%%%%%%%%%%%%%%%%%%%%%%%%%%%
\titreactivite{Ramsey graphs and combinatorics}

\descriptionactivite{You will see that a very simple problem, which concerns the relationships between only six people, will require a lot of calculations to be solved.}

%%%%%%%%%%%%%%%%%%%%%%%%%%%%%%%%%%%%%%%%%%%%%%%%%%%%
\titreactivite{Bitcoin}

\descriptionactivite{The \emph{bitcoin} is a dematerialized and decentralized currency. It is based on two computer principles: public key cryptography and proof of work. To understand this second principle, you will create a simple model of bitcoin.}

%%%%%%%%%%%%%%%%%%%%%%%%%%%%%%%%%%%%%%%%%%%%%%%%%%%%
\titreactivite{Random blocks}

\descriptionactivite{You will program two methods to build figures that look like algae or corals. Each figure is made up of small randomly thrown blocks that stick together.}


